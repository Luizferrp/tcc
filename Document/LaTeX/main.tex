\documentclass[conference]{IEEEtran}
\IEEEoverridecommandlockouts
% The preceding line is only needed to identify funding in the first footnote. If that is unneeded, please comment it out.
\usepackage{cite}
\usepackage{amsmath,amssymb,amsfonts}
\usepackage{algorithmic}
\usepackage{graphicx}
\usepackage{textcomp}
\usepackage{xcolor}

\DeclareUnicodeCharacter{1EBD}{\~{e}}
\def\BibTeX{{\rm B\kern-.05em{\sc i\kern-.025em b}\kern-.08em
    T\kern-.1667em\lower.7ex\kern-.125emX}}
\begin{document}

\title{Modelos de Aprendizagem de Máquina para Sistemas de Detecção de Intrusão\\
{\footnotesize \textsuperscript{}}{da Industria 4.0 }
}
\author{
\IEEEauthorblockN{1\textsuperscript{th} Luiz Fernando Reis Pereira}
\IEEEauthorblockA{\textit{Bachelor of Computer Science} \\
\textit{Pontifical Catholic University of Paraná}\\
Curitiba, Brasil \\
0009-0008-3781-0134}
\and
\IEEEauthorblockN{2\textsuperscript{st} Henrique Conceição ta bom ja}
\IEEEauthorblockA{\textit{Bachelor of Computer Science} \\
\textit{Pontifical Catholic University of Paraná}\\
Curitiba, Brazil \\
email@pucpr.edu.br}
\and
\IEEEauthorblockN{3\textsuperscript{nd} Eduarda Dallagrana}
\IEEEauthorblockA{\textit{Bachelor of Computer Science} \\
\textit{Pontifical Catholic University of Paraná}\\
Curitiba, Brazil \\
email@pucpr.edu.br}
\and
\IEEEauthorblockN{4\textsuperscript{rd} Kauã Nunes}
\IEEEauthorblockA{\textit{Bachelor of Computer Science} \\
\textit{Pontifical Catholic University of Paraná}\\
Curitiba, Brazil \\
email@pucpr.edu.br}
}

\maketitle

\begin{abstract}
Com a popularização da internet e alta disponibilidade graças a redes sem fios e dispositivos iot, desde nossos dados mais triviais como uma lista de compras a nossos dados pessoais de identificação e posses se encontram, em algum grau, expostos as redes. Convulutivamente mais dados foram se tornando fáceis de se conseguir, começando uma corriga tecnologica de ataque e proteção. ataques, principalmente zero day, não podem ser fácilmente diagnosticado estocasticamente, então, qual o estado da arte da performace de detcção de ataques para os dispositivos modernos? e quais as técnicas de performace promissoras para a proxima geração de sistema de cybersegurança ?

\end{abstract}

\begin{IEEEkeywords}
Network Intrusion and Anomaly Detection (NID), Zero-day Attack Detection, Attack Classification, Feature Engineering for NID, Deep Learning and Neural Networks for Cybersecurity, Feature Selection in NID, Transfer Learning for NID, Federated Learning for Intrusion Detection
\end{IEEEkeywords}

\section{Introdução}

As redes permitiram que grandes quantidades de conteúdos fossem trafegados, juntamente com a portabilidade dos dispositivos que podem se conectar com elas, principalmente IOTs e que cada vez mais não só são mais populares e mais presentes - como em vestíveis ou soluções de automação - como são essenciais e diários na vida de todos, muitas vezes carregando informações pessoais, ou apenas capacidade de hardware suficiente para chamar atenção de agentes maliciosos, se tornando alvos de ataques. Ataques previsíveis podem ser facilmente mitigados por uma equipe dedicada, mas ataques particulares, em especial tipo day zero, se tornam impossíveis de evitar, mas ainda podem apresentar características de tráfego similares, seja para se conectar com servidores suspeitos ou tráfego fora do comum; O que nos traz a possibilidade de usar um modelo de aprendizagem de máquina para treinar reconhecimento desses padrões, em especial Deep learning pra apresentar o estado da arte em aprendizado

\section{Reletade Work}

\subsection{Network Intrusion Analises}

\subsection{Network Machine Learning Aproach}

\subsection{Relevância}

\subsection{Métricas de Desempenho}

\textbf{Matriz de Confusão:} A matriz de confusão fornece uma visão detalhada da performance do modelo, mostrando como ele classificou corretamente ou incorretamente as instâncias em cada classe (baixo, médio e alto consumo). A matriz identifica o número de verdadeiros positivos, falsos positivos, verdadeiros negativos e falsos negativos, o que permite visualizar de forma clara onde o modelo acerta e onde comete erros [18]. \\
    
\textbf{Precisão (Precision):} A precisão mede a proporção de previsões corretas entre todas as instâncias classificadas como pertencentes a uma determinada classe. No caso de "alto consumo", por exemplo, a precisão indica o quão bem o modelo classificou os períodos de alto consumo corretamente, evitando falsos positivos. Para este estudo, uma alta precisão é importante para garantir que as previsões de alto consumo sejam confiáveis [18]. \\

    \begin{math} Precisao = \frac{VP}{VP + FP}
    \end{math}\\

Onde: 

VP (Verdadeiros Positivos): Número de previsões corretas. 

FP (Falsos Positivos): Número de previsões incorretas em que o modelo classificou erroneamente uma instância como pertencente à classe de "alto consumo". \\
    
\textbf{Recall (Sensibilidade ou TPR - True Positive Rate):} O recall mede a capacidade do modelo de identificar corretamente todas as instâncias de uma determinada classe. Ele responde à pergunta: "das instâncias que realmente pertencem à classe 'alto consumo', quantas foram corretamente identificadas?". Um alto recall é importante para garantir que o modelo não deixe de identificar períodos críticos de alto consumo [18]. \\

    \begin{math} Recall = \frac{VP}{VP + FN}
    \end{math} \\
    
Onde: 
    
FN (Falsos Negativos): Número de previsões incorretas em que o modelo não identificou corretamente uma instância de "alto consumo". \\
    
\textbf{F1-score:} O F1-score é a média harmônica entre a precisão e o recall. Ele é útil quando se deseja um equilíbrio entre os dois, especialmente quando há um trade-off entre a capacidade de evitar falsos positivos (alta precisão) e a capacidade de identificar todas as instâncias reais (alto recall) [18]. 
    
O F1-score é especialmente relevante quando há um desequilíbrio entre as classes ou quando tanto a precisão quanto o recall são igualmente importantes, como no caso da identificação de picos e anomalias no consumo energético.[15] \\\\
    
    \begin{math} F1 = 2\cdot \frac{Precisao \cdot Recall}{Precisao + Recall}
    \end{math} \\


\section{Materizalização}
\subsection{Materiais e Métodos}

Para o desenvolvimento deste projeto, foi utilizada uma abordagem baseada em machine learning para a classificação do consumo energético em uma empresa metalúrgica, especificamente com a aplicação do algoritmo \textit{Random Forest Classifier}. Para isso, foram empregados diversos recursos tecnológicos e ferramentas, que serão descritos a seguir.  

O ambiente de desenvolvimento foi configurado em uma máquina virtual rodando o sistema operacional Ubuntu 22, que oferece estabilidade e robustez, sendo ideal para o processamento de grandes volumes de dados e a execução eficiente de algoritmos de machine learning. 

A linguagem de programação utilizada foi Python, amplamente empregada em ciência de dados devido à sua vasta gama de bibliotecas especializadas. No contexto deste estudo, a biblioteca pandas foi utilizada para a manipulação dos dados tabulares e para o pré-processamento do arquivo CSV que continha o conjunto de dados. Além disso, a biblioteca NumPy foi utilizada para operações matemáticas e manipulação de arrays numéricos, essenciais para o suporte ao processamento dos dados. 
O algoritmo de classificação foi implementado utilizando a biblioteca scikit-learn, que ofereceu as ferramentas necessárias para a construção do modelo \textit{Random Forest Classifier} e para a validação dos resultados, escolhida por sua ampla variedade de algoritmos de aprendizado de máquina e sua eficiência na modelagem e processamento de dados, permitindo a implementação e o teste dos modelos de forma ágil e eficaz. Além disso, para a visualização dos resultados, foi utilizada a biblioteca matplotlib, que possibilitou a criação de gráficos, como a matriz de confusão, fornecendo uma análise visual das classificações e dos erros cometidos pelo modelo. 

A base de dados utilizada neste estudo foi obtida da plataforma Kaggle, consistindo em um conjunto de dados históricos de consumo de energia de uma empresa metalúrgica. O arquivo, no formato CSV, incluía informações como data, consumo de energia em kilowatt-hora (kWh) e tipo de carga. Esses dados foram cruciais para a construção dos modelos de classificação, pois permitiram a análise e identificação de padrões de consumo energético ao longo do tempo. Segue a relação de colunas existentes na base com a descrição de cada coluna feita pelo autor e uma imagem com exemplos de dados (Fig. 1):

\textit{\begin{itemize}
    \item Date: Continuous-time data taken on the first of the month 
    \item Usage kWh: Industry Energy Consumption Continuous kWh 
    \item Lagging Current: reactive power Continuous kVarh 
    \item Leading Current: reactive power Continuous kVarh 
    \item CO2: Continuous ppm 
    \item NSM: Number of Seconds from midnight Continuous S 
    \item Week status Categorical (Weekend (0) or a Weekday(1)) 
    \item Day of week: Categorical Sunday, Monday : Saturday 
    \item Load Type: Categorical Light Load, Medium Load, Maximum Load
\end{itemize}}

\begin{figure}[ht]
    \centering
%    \includegraphics[width=1\linewidth]{tabela base.png}
    \caption{Exemplo de 10 registros da base de dados. Imagem do Autor}
    \label{fig:enter-label3}
\end{figure}

\subsection{Métodos}

A metodologia aplicada neste estudo centrou-se no desenvolvimento de um modelo de (aprendizado supervisionado) para a classificação de padrões de consumo de energia em uma empresa metalúrgica. O objetivo principal foi classificar o consumo de energia em diferentes níveis, identificar picos de consumo, detectar anomalias e fornecer insights úteis para a gestão energética da empresa. A seguir, detalhamos as etapas do processo, desde a preparação dos dados até a avaliação do desempenho do modelo. 


\begin{enumerate}
    \item \textbf{Coleta e Preparação dos Dados:} 
    pass
    
    \item \textbf{Classificação de Anomalias:} Anomalias foram definidas como desvios significativos nos padrões normais de consumo. A detecção dessas anomalias forneceu uma base para identificar problemas potenciais, como  

    \item \textbf{algoritimos \& modelos: }
    pass
        
    \item \textbf{Construção do Modelo:} O modelo selecionado para este estudo foi o Random Forest Classifier, reconhecido por sua robustez e capacidade de lidar com variáveis complexas e interações não lineares, além de sua eficácia em problemas de classificação e regressão [11]. O Random Forest é um algoritmo de ensemble, composto por múltiplas árvores de decisão independentes. Cada árvore realiza previsões individuais e, posteriormente, as previsões finais são feitas com base em uma votação majoritária entre as árvores. Essa abordagem contribui para o aumento da precisão do modelo, ao mesmo tempo que reduz o risco de overfitting, o que a torna uma das técnicas mais amplamente utilizadas em cenários industriais [12]. 
    
    \item \textbf{Avaliação de Desempenho:} A avaliação do modelo foi realizada utilizando o conjunto de teste. A performance foi medida por meio de três métricas principais: precisão, recall e F1-score, além da matriz de confusão, para garantir que as previsões fossem consistentes com os padrões de consumo reais da empresa.\\
    
\end{enumerate}

\begin{thebibliography}{00}

\bibitem{b1}Grzegorz Kinelski, ‘‘The main factors of successful project management in the aspect of energy enterprises’ efficiency in the digital economy environment’’ POLITYKA ENERGETYCZNA – ENERGY POLICY JOURNAL,Volume 23, no. 3, pp. 5–20, 2020.
\end{thebibliography}

\end{document}
