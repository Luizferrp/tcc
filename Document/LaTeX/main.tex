\documentclass[conference]{IEEEtran}
\IEEEoverridecommandlockouts
% The preceding line is only needed to identify funding in the first footnote. If that is unneeded, please comment it out.
\usepackage[utf8]{inputenc}
\usepackage[T1]{fontenc}
\usepackage{cite}
\usepackage{amsmath,amssymb,amsfonts}
\usepackage{algorithmic}
\usepackage{graphicx}
\usepackage{textcomp}
\usepackage{xcolor}
\usepackage[portuguese]{babel}
\usepackage[colorlinks=true,allcolors=black]{hyperref}
\usepackage[acronym]{glossaries}

% não permite separar acrônimo em duas linhas (só a sigla estar na outra linha)
\newacronymstyle{long-short-br}
{%
  \GlsUseAcrEntryDispStyle{long-short}%
}%
{%
  \GlsUseAcrStyleDefs{long-short}%  
  \renewcommand*{\genacrfullformat}[2]{%
    \glsentrylong{##1}##2~\textup{(\firstacronymfont{\glsentryshort{##1}})}%
  }%
  \renewcommand*{\Genacrfullformat}[2]{%
    \Glsentrylong{##1}##2~\textup{(\firstacronymfont{\glsentryshort{##1}})}%
  }%
  \renewcommand*{\genplacrfullformat}[2]{%
    \glsentrylongpl{##1}##2~\textup{(\firstacronymfont{\glsentryshortpl{##1}})}%
  }%
  \renewcommand*{\Genplacrfullformat}[2]{%
    \Glsentrylongpl{##1}##2~\textup{(\firstacronymfont{\Glsentryshortpl{##1}})}%
  }%
}
\setacronymstyle{long-short-br}


\usepackage{graphicx}
\usepackage{xcolor}
\usepackage{colortbl}
\usepackage{array}
\usepackage{booktabs} % Melhora a aparência das tabelas
\usepackage{soul}
\setuldepth{Berlin}


\definecolor{headercolor}{RGB}{220,220,220} % Cinza claro para cabeçalhos
\definecolor{rowcolor}{RGB}{245,245,245} % Cinza ainda mais claro para linhas alternadas
\newcommand\red[1]{{\textcolor{red}{#1}}}

\sloppy
\begin{document}

\input{acronyms} % importa os acrônimos de acronyms.tex

\title{Revisão da Literatura\\
\vspace{0.4cm}
Tema de pesquisa: Detecção de Intrusão de Redes em CNNs para hardwares limitados\\
\vspace{0.4cm}
{\Large Projeto de Pesquisa Científica\\
Experiência Criativa: Projeto Transformador I\\
Bacharelado em Ciência da Computação -- PUCPR\\
2025}}

\author{
    Turma U -- Equipe 03 \\
    Eduarda Dallagrana,  Luiz Pereira, Kaua Nunes, Henrique conceição \\
    {\tt\small \red{dudadallagrana@gmail.com}, 0009-0008-3781-0134, \red{email3}, henrique.conceicao19.hc@gmail.com}
}
\maketitle


\section{Fundamentação Teórica}

A crescente conectividade expõe novos riscos á toda informação cada vez mais abrangente e sensível, ampliando as oportunidades para ataques cibernéticos. exigindo mecanismos de proteção eficazes. No entanto, a evolução dos ataques, incluindo exploits customizados e ameaças "zero-day", desafia a detecção e resposta rápida, tornando essencial o aprimoramento das estratégias de segurança de forma performática.

Abordagens tradicionais de detecção de intrusão, como árvores de decisão e k-Nearest Neighbors (KNN), apresentam limitações em termos de tempo de processamento e escalabilidade. Entretanto, tecnicas e algoritimos de apreendizarem profunda surgem como uma alternativa promissora, ajustando automaticamente os pesos das entradas sem necessidade de armazenar grandes volumes de dados no processo de identificação. Porém, esses modelos ainda enfrentam desafios computacionais, especialmente quando executados em hardware de uso geral.

Com o crescimento exponencial dos dispositivos de processamento limitado, Iot e a dependência crescente de software de terceiros, o desenvolvimento de sistemas de detecção de intrusão mais rápidos e eficientes são a única defesa dos usuários contra ataques. O desenvolvimento uma solução de alta performace e baixo consumo de hardware capaz de detectar ameaças em tempo real, protegendo tanto dispositivos limitados quanto redes públicas, evita que dados sensíveis de se tornarem prejuízo.

\subsection{Sistemas de Detecção de Intrusão}

    O Instituto Nacional de Padrões e Tecnologia (NIST) define uma intrusão como qualquer ação cujo objetivo seja comprometer a confidencialidade, integridade e disponibilidade de um dispositivo, sistema ou rede, além de violar seus mecanismos de segurança (BACE; MELL, 2001). Dessa forma, um Sistema de Detecção de Intrusão (Intrusion Detection System – IDS) pode ser descrito como um conjunto de ferramentas automatizadas responsáveis por monitorar e analisar eventos em redes ou sistemas computacionais, buscando identificar possíveis atividades maliciosas.
    
    Os IDSs são categorizados em dois tipos principais: (i) Sistema de Detecção de Intrusão em Hospedeiro (Host Intrusion Detection System – HIDS) e (ii) Sistema de Detecção de Intrusão em Rede (Network Intrusion Detection System – NIDS). O HIDS tem como função analisar comportamentos específicos nos dispositivos finais, como o acesso a arquivos, uso de aplicativos e registros de logs. Já o NIDS foca na análise do tráfego de rede, procurando padrões suspeitos de comportamento. Normalmente, um NIDS é posicionado estrategicamente, como próximo a firewalls ou em segmentos críticos da rede (SILVA et al., 2022).
    
    Há três abordagens principais para detecção de intrusão. A detecção baseada em assinatura identifica ataques conhecidos ao compará-los com padrões previamente registrados. Já a detecção por anomalia considera qualquer desvio em relação a um comportamento previamente estabelecido como normal. Diferente da primeira abordagem, essa metodologia permite detectar ameaças ainda desconhecidas. Por fim, a análise de protocolo com estado examina a relação entre requisições e respostas dentro dos protocolos da rede monitorada, permitindo identificar sequências inesperadas e, consequentemente, possíveis intrusões desconhecidas.


\subsection{Aprendizagem de máquina}

    O Aprendizado de Máquina é um processo automatizado onde algoritmos de computador aprendem a identificar padrões em dados (Kelleher et al., 2015). Essa área integra conhecimentos de matemática, estatística e ciência da computação, permitindo análises preditivas avançadas, conforme destacado por Muller e Guido (2017) e Aparicio (2019). Em termos de desempenho dos algoritmos, é importante considerar dois fatores fundamentais: viés e variância. O viés refere-se ao erro introduzido pela simplificação excessiva do modelo, enquanto a variância reflete a sensibilidade do modelo a pequenas variações nos dados de treinamento. Um modelo ideal busca o equilíbrio entre ambos para evitar problemas de underfitting (viés alto) e overfitting (variância alta), garantindo generalização adequada aos dados.

    As tarefas de aprendizado de máquina são tipicamente classificadas em três categorias: aprendizado supervisionado, não supervisionado e semi-supervisionado. No aprendizado supervisionado, o algoritmo aprende padrões a partir de exemplos de entradas e saídas desejadas, fornecidos por um especialista. Assim, ele adquire a capacidade de mapear entradas válidas, de um mesmo contexto, para uma saída correspondente.
    
    Entre as arquiteturas de redes neurais profundas, a Convolutional Neural Network (CNN) se destaca por utilizar camadas de convolução para extrair automaticamente características relevantes do tráfego de rede. Cada camada de convolução aplica filtros (kernels) aos dados de entrada, permitindo identificar padrões anômalos e assinaturas de ataques cibernéticos.

    As camadas de pooling, como Max Pooling e Average Pooling, seguem as camadas de convolução e têm a função de reduzir a dimensionalidade dos mapas de características, diminuindo assim a carga computacional e ajudando a evitar o overfitting. O Max Pooling seleciona o valor máximo dentro de uma região do mapa de características, enquanto o Average Pooling calcula a média dos valores, preservando informações essenciais.
    
    Após a extração de características, as camadas flatten transformam os mapas multidimensionais em um vetor unidimensional, que é então alimentado nas camadas totalmente conectadas (fully connected layers). Essas camadas finais realizam a classificação, diferenciando tráfego legítimo de possíveis tentativas de intrusão com base nos padrões aprendidos.
    
    As CNNs têm se tornado uma abordagem promissora para a detecção de intrusão em redes, devido à sua capacidade de analisar grandes volumes de dados e identificar ameaças de forma automática e eficiente.

\subsection{apreendizagem profunda de maquina}

\subsection{tecnicas de avaliação}

\subsection{tecnicas de refinamento classicas}


\section{Trabalhos Relacionados}

\subsection{metodologia}

  As palavras chaves para busca dos artigos se basearam em 

\subsection{Revisão}


    \cite{1} realiza a comparação entre diversas abordagens de aprendizado de máquina, diferentes misturas de features para os algoritmos de classificação, abordando as  vantagens e desvantagens do uso de cada uma das técnicas e ferramentas, realizando um estudo
    comparativo da abordagem de classificadores

    \cite{2} Realiza um estudo de sistemas de detecção de intrusão em redes utilizando artigos e conhecimentos recentes como o progresso do uso de Aprendizado de máquina e inteligência artificial no campo dos sistemas de detecção de intrusão, bem como uma análise criteriosa de performance, bases utilizadas e precisão.

    \cite{3} O artigo mostra os desafios modernos de NIDS como:

    Volume de Dados: A crescente quantidade de dados trafegando nas redes torna a detecção de intrusões mais complexa e requer mais velocidade para continuidade da operação dos sistemas impondo-se barreiras entre velocidade e acurácia nas técnicas convencionais de análise.
    
    Precisão: As técnicas tradicionais requerem alto grau de intervenção humana ao considerarem usuos de modelos rasos principalmente ao se basearem em outliers, causando muitos falsos negativos
    
    Diversidade e Dinamismo: A crescente variedade de dispositivos e protocolos torna desafiador diferenciar tráfego normal de anormal. Ademais, as redes são dinâmicas e mutáveis, dificultando a criação de padrões estáticos para identificação de ameaças.
    
    Ataques de Baixa Frequência e zero day: Esses ataques são particularmente desafiadores para técnicas tradicionais devido ao desbalanceamento ou inexistencia dos datasets de treinamento.

    e limitações de pesquisas como: 
            
    Métodos de "shallow classifiers" de detecção de intrusão possuem muitas limitações com dessas abordagens que acabam demandando grande envolvimento de especialistas humanos para interpretação de dados, identificação de padrões e ajustes de regras, limitando a escalabilidade e agilidade na detecção de novas ameaças.
        
    O deep learning (DL) tem se destacado como uma solução avançada para modelagem de relações complexas e representações mais abstratas de dados. Dentro desse campo, autoencoders se mostram promissores ao permitirem extração de características relevantes de forma não-linear, sem necessidade de rótulos. Isso facilita a identificação de anomalias sem um conjunto fixo de assinaturas predefinidas.
        
    Para superar as limitações das abordagens clássicas e do deep learning isoladamente, o artigo propos 
    Autoencoders de Deep Learning: Utilizados para capturar a dinâmica dos padrões de tráfego de rede, reduzindo a necessidade de interação humana e aumentando a capacidade de interpretação genérica.
    
    Random Forest: Aplicado diretamente sobre as saídas do autoencoder, sem a necessidade de um layout achatado (flatten), garantindo maior preservação da estrutura dos dados e melhor capacidade de classificação.
    
    Essa abordagem híbrida busca aliar a robustez do deep learning à alta acurácia do Random Forest, criando um sistema capaz de lidar com a dinamicidade das redes modernas sem comprometer a confiabilidade na detecção de intrusões.

    \cite{4} A segurança cibernética enfrenta desafios crescentes devido à explosão no volume e na variedade de ataques. O aumento no número de ataques zero-day torna os sistemas de detecção de intrusão (IDS) baseados em assinaturas menos eficazes, pois dependem de bancos de dados com históricos de ataques conhecidos. Para enfrentar essas limitações, propõe-se uma abordagem baseada na combinação de autoencoders e Support Vector Machine (SVM).

Desafios Atuais na Detecção de Ataques Zero-Day

A dependência de assinaturas históricas limita a detecção de ataques novos e desconhecidos.

Métodos convencionais de detecção de outliers apresentam altas taxas de falso negativo.

Estudos indicam que ataques zero-day podem permanecer ativos por um médio de 10 meses antes de serem detectados, comprometendo sistemas durante esse período.

Estatísticas mostram que 62\% dos ataques só são identificados após comprometerem os sistemas.

Em 2019, o número de ataques zero-day superou os três anos anteriores, evidenciando a necessidade urgente de modelos mais eficazes.

Proposta de Solução: Autoencoder + SVM

A solução proposta explora os pontos fortes do deep learning e da SVM para melhorar a detecção de ataques zero-day:

Autoencoder: Atua como um detector leve de outliers, capturando padrões de tráfego normais e identificando desvios, o que permite detectar ataques desconhecidos com alta taxa de recall.

SVM: Utilizado para classificar os padrões extraídos pelo autoencoder sem a necessidade de um layout achatado (flatten), garantindo maior preservação da estrutura dos dados e melhor capacidade de classificação.

Benefícios da Abordagem

Redução da dependência de especialistas humanos para ajustes constantes.

Maior capacidade de adaptação a novas ameaças sem necessária reconfiguração manual.

Melhor detecção de ataques de baixa frequência e zero-day, que passam despercebidos por abordagens tradicionais.

A combinação de autoencoders e SVM representa um passo significativo para sistemas de detecção de intrusão mais eficientes e adaptáveis aos desafios das redes modernas.

    \cite{5} **Revisão de Literatura: Otimização do Treinamento de Redes Neurais**

    O treinamento de redes neurais profundas enfrenta desafios significativos devido à natureza complexa da superfície de custo, que é tipicamente não-quadrática, não-convexa e de alta dimensionalidade. Como resultado, o algoritmo de backpropagation pode ser lento e não há garantias de que a convergência para uma solução adequada ocorra de maneira rápida ou sequer aconteça (Sompolinsky et al., 37; Darken & Moody, 9; Sutton, 38; Murata et al., 28). Para mitigar esses problemas, diversas técnicas foram propostas, incluindo taxas de aprendizado adaptativas e early stopping.
    
    ### Taxas de Aprendizado Adaptativas
    
    As taxas de aprendizado desempenham um papel crucial na eficiência do treinamento de redes neurais. Diferentes abordagens foram propostas para ajustar automaticamente a taxa de aprendizado com base no erro, de modo a controlar a velocidade de convergência. Uma das estratégias sugeridas é equalizar a velocidade de aprendizado, permitindo que cada peso possua sua própria taxa de aprendizado, proporcional à raiz quadrada do número de entradas da unidade correspondente. Além disso, é recomendado que os pesos das camadas inferiores sejam maiores do que os das camadas superiores para melhor estabilização do treinamento.
    
    ### Early Stopping como Mecanismo de Regularização
    
    O early stopping é uma técnica amplamente utilizada para evitar o overfitting em redes neurais. O processo consiste em interromper o treinamento assim que o erro no conjunto de validação começa a aumentar, impedindo que o modelo se ajuste excessivamente aos dados de treinamento. Os principais passos do early stopping incluem:
    
    1. **Divisão dos Dados:** O conjunto de treinamento é separado em subconjuntos de treino e validação, frequentemente utilizando uma proporção de 2:1.
    2. **Monitoramento Periódico:** O modelo é treinado apenas com os dados de treino, enquanto a perda no conjunto de validação é verificada periodicamente, por exemplo, a cada cinco épocas.
    3. **Identificação do Ponto Ideal:** Se a perda de validação começar a aumentar em relação à última medição, o treinamento é interrompido.
    4. **Utilização dos Melhores Pesos:** Os pesos da rede no momento imediatamente anterior ao aumento da perda de validação são utilizados como a melhor representação do modelo.
    
    O early stopping assume que a perda de validação é um indicador da capacidade de generalização do modelo, ajudando a melhorar o desempenho em dados não vistos (Why Early Stopping?).
    
    ### Conclusão
    
    Diante dos desafios inerentes ao treinamento de redes neurais, a otimização do processo é essencial para garantir modelos eficazes e eficientes. O uso de taxas de aprendizado adaptativas e early stopping demonstrou ser uma abordagem promissora para mitigar problemas como a convergência lenta e o overfitting. Assim, a seleção criteriosa dessas técnicas pode levar a modelos mais robustos e com melhor capacidade de generalização.
        

    \cite{6}
    \cite{7}
    \cite{8}
    \cite{9}
    \cite{10}
    \cite{11}
    \cite{12}
    \cite{13}
    %sdn
    \cite{14} Sistemas de Detecção de Intrusão em Redes (NIDS), Aprendizado de Máquina (ML) e Aprofundado (DL) e Redes Definidas por Software (SDN).
    • NIDS: Sistemas projetados para detectar atividades maliciosas em redes, como vírus, worms e ataques DDoS. Podem ser baseados em assinatura (detecção de ameaças conhecidas) ou anomalias (detecção de comportamentos incomuns).
    • ML/DL: Técnicas como SVM, Random Forest, Redes Neurais Convolucionais (CNN) e Autoencoders são aplicadas para melhorar a precisão e reduzir falsos positivos em NIDS.
    • SDN: Arquitetura que separa os planos de controle e dados, permitindo programabilidade e gerenciamento centralizado. Essa flexibilidade é ideal para implementar NIDS dinâmicos e escaláveis.
    A revisão evidenciou que a combinação de SDN, ML e DL oferece vantagens significativas para NIDS, como maior precisão e adaptabilidade. No entanto, desafios persistem, como a otimização de recursos computacionais e a criação de datasets mais representativos. Como direção futura, propos investigar:
    • Técnicas híbridas (ex.: SOM + Redes Neurais Recorrentes).
    • Implementação em infraestruturas críticas (ex.: data centers).
    %survey current network
    \cite{15} Fundamentação Teórica
    Os Sistemas de Detecção de Intrusão em Redes (NIDS) são essenciais para a segurança cibernética, atuando na identificação de atividades maliciosas em tempo real. Eles se dividem em duas abordagens principais:
        1. Baseada em Assinaturas: Compara o tráfego com um banco de dados de padrões conhecidos de ataques (ex.: vírus, worms). É eficaz contra ameaças conhecidas, mas falha contra ataques novos. Algoritmos como Aho-Corasick e expressões regulares são comumente usados para inspeção profunda de pacotes.
        2. Baseada em Anomalias: Cria um perfil do tráfego "normal" e alerta sobre desvios. Técnicas incluem métodos estatísticos, aprendizado de máquina (ex.: redes bayesianas) e mineração de dados. Apesar de promissora, essa abordagem sofre com altas taxas de falsos positivos.
        O estado da arte em NIDS evoluiu para combinar técnicas reativas (assinaturas) e proativas (anomalias), porém persistem desafios:
        • Limitações: Falsos positivos, desempenho em redes de alta velocidade e criptografia.
        • Futuro: Tendência para sistemas distribuídos e autônomos, como NIPS (Prevenção de Intrusão).
    Abordagem metodológica sugerida:
        • Quantitativa: Avaliação de métricas de desempenho (ex.: throughput, taxa de detecção).
        • Qualitativa: Análise de falsos positivos/negativos em datasets reais (ex.: CICIDS2017).
    %performace
    \cite{16} A análise de sentimentos é uma aplicação crítica de NLP, tradicionalmente baseada em métodos como Bag of Words (BoW) e lexicons, que exigem extração manual de features. Com o avanço do deep learning, técnicas como Word2Vec, GloVe e modelos de ensemble (e.g., NsGA) permitem extração automática de features e melhor desempenho. O artigo propõe um modelo híbrido que combina features superficiais (handcrafted) e profundas (deep learning), superando limitações de generalização e especialização. Segundo Pandian (2021), a integração de vetores de palavras afetivas (NA), genéricas (NG) e superficiais (Ns) resulta em uma melhoria significativa na precisão de classificação (até 95\% F-Score em alguns datasets).


    Principais lacunas identificadas:
    1. Dependência de grandes volumes de dados para modelos de deep learning.
    2. Complexidade computacional em ensembles (e.g., NsGA demanda 980s para treinamento).

    Considerações Finais
O estudo demonstra que modelos híbridos (e.g., NsGA) superam abordagens tradicionais em análise de sentimentos, mas exigem otimização para reduzir custos computacionais. Como futuros trabalhos, sugere-se:
    • Aplicação em outros idiomas (e.g., português).
    • Uso de técnicas de compressão de modelos (e.g., quantização) para viabilizar implantação em dispositivos edge.

\begin{table}[!htb]
    \centering
    \setlength{\tabcolsep}{1pt}
    \caption{Distribuição do número de artigos de conferência e de periódico por estrato Qualis.}
    \label{tab:sumario-qualis}
    
    \vspace{1mm}
    
    \begin{tabular}{ccc}
    \toprule
    \begin{tabular}[c]{c}\textbf{Estrato}\\\textbf{Qualis}\end{tabular} & \begin{tabular}[c]{c}\textbf{Artigos de}\\\textbf{Conferência}\end{tabular} & \begin{tabular}[c]{c}\textbf{Artigos de}\\\textbf{Periódico}\end{tabular} \\ \midrule
    A1     & $0$           & $0$ \\
    A2     & $0$           & $0$ \\
    A3     & $0$           & $0$ \\
    A4     & $0$           & $0$ \\
    Outro  & $0$           & $0$ \\ \bottomrule        
    \end{tabular}
\end{table}



\begin{table}[!htb]
    \centering
    \setlength{\tabcolsep}{4pt}
    \caption{Distribuição do número de artigos por período de publicação.}
    \label{tab:sumario-periodo}
    
    \vspace{1mm}
    
    \begin{tabular}{cc}
    \toprule
    \begin{tabular}[c]{c}\textbf{Ano de}\\\textbf{Publicação}\end{tabular} & \textbf{Artigos} \\ \midrule
    $2021$ - $2025$     & $0$ \\
    $2016$ - $2020$     & $2$ \\
    $2011$ - $2015$     & $0$ \\
    $1900$ - $2010$     & $0$ \\ \bottomrule        
    \end{tabular}
\end{table}


  \begin{table*}[!htb]
    \centering
    \caption{Comparação de Trabalhos no Estado-da-Arte}
    \renewcommand{\arraystretch}{1.3} % Ajuste de espaçamento entre linhas
    \resizebox{\textwidth}{!}{ % Ajusta automaticamente a largura da tabela
        \begin{tabular}{ccccccc}
            \toprule
            \textbf{Autor(es) / Ano} & \textbf{Tema Principal} & \textbf{Método} & \textbf{Base / Amostra} & \textbf{Resultados} & \textbf{Protocolo / Ferramentas} & \textbf{Crítica e Limitações} \\
            \midrule
           Biswas (2018) & Comparação de classificadores para detecção de intrusão &Algoritmos de aprendizado de máquina (Random Forest, SVM, KNN, etc.) & Dataset NSL-KDD & Random Forest: 89.83\% SVM: 82.23\% KNN: 84.16\% de acurácia & Weka e Python & Desempenho variável entre classificadores; necessidade de otimização de features \\  
           Ahmad et al. (2020) & Uso de ML e IA em sistemas de detecção de intrusão & Revisão sistemática e análise de performance & KDD, CICIDS, UNSW-NB15 & Varia entre 85\% e 99\% de acurácia dependendo do modelo e dataset & Análise comparativa de artigos científicos & Pouca generalização e falta de padronização na avaliação dos modelos \\   
            \bottomrule
        \end{tabular}
    }
    
    \label{tab:sota-overview}
    \end{table*}
    

\section{Considerações Finais}


    Este estudo analisou a detecção de intrusão em redes usando técnicas de aprendizado profundo, com foco em Redes Neurais Convolucionais (CNNs). A pesquisa mostrou que métodos tradicionais, como os baseados em assinaturas, têm limitações na identificação de ataques desconhecidos e requerem intervenção manual na extração de características. As CNNs, por outro lado, podem automatizar a extração de padrões importantes dos dados de tráfego, melhorando a precisão e a eficiência dos sistemas.
    
    Os benefícios das CNNs incluem aprendizado contínuo e capacidade de identificar ameaças complexas. Contudo, desafios como alto custo computacional e necessidade de muitos dados rotulados ainda dificultam sua adoção ampla. A maioria dos estudos usa conjuntos de dados específicos, o que pode limitar a aplicabilidade em diferentes cenários.
    
    Futuros trabalhos podem combinar CNNs com outras técnicas de aprendizado de máquina para melhorar a detecção e reduzir falsos positivos. Também, abordagens de aprendizado semi-supervisionado ou auto-supervisionado podem diminuir a dependência de dados rotulados.
    
    Para desenvolver esta pesquisa, a abordagem será quantitativa, avaliando a performance das CNNs através de métricas como acurácia e taxa de falsos alarmes, e realizando estudos de caso com bases de dados variadas. O objetivo é contribuir para a detecção inteligente de ameaças e aumentar a eficiência e adaptabilidade dos sistemas de segurança cibernética.
    
\begin{thebibliography}{00}

\bibitem{b1}BISWAS, Saroj Kr. Intrusion Detection Using Machine Learning: A Comparison Study. 2018. NIT Silchar.
\bibitem{b2}AHMAD, Zeeshan et al. Network intrusion detection system: A systematic study of machine learning and deep learning approaches. 2020. UniversitiMalaysia Sarawak.
\bibitem{b3} Nathan Shone; Tran Nguyen Ngoc; Vu Dinh Phai; Qi Shi  (February 2018), ''A Deep Learning Approach to Network Intrusion Detection'', Australian Journal of Mechanical Engineering, IEEE Transactions on Emerging Topics in Computational Intelligence ( Volume: 2, Issue: 1, February 2018)
\bibitem{b4} Springer, Berlin, Heidelberg, ''Early Stopping - But When?'' Part of the book series: Lecture Notes in Computer Science ((LNCS,volume 1524))
\bibitem{b5} Hanan Hindy,Robert Atkinson, Christos Tachtatzis,Jean-Noël Colin, Ethan Bayne, and Xavier Bellekens, (14 October 2020), ''Utilising Deep Learning Techniques for Effective Zero-Day Attack Detection'' ELECTRONIC COMMERCE RESEARCH AND APPLICATIONS, doi: https://doi.org/10.3390/electronics9101684
\bibitem{b6} Akinul Islam Jony, Arjun Kumar Bose Arnob, (August 8, 2024), ''Securing the Internet of Things: Evaluating Machine Learning Algorithms for Detecting IoT Cyberattacks Using CIC-IoT2023 Dataset'', DOI: 10.5815/ijitcs.2024.04.04
\bibitem{b7} , , ''Robust Botnet Detection Approach for Known and Unknown Attacks in IoT Networks Using Stacked Multi-classifier and Adaptive Thresholding''
\bibitem{b8} , , ''Securing the Internet of Things: Evaluating Machine Learning Algorithms for Detecting IoT Cyberattacks Using CIC-IoT2023 Dataset''
\bibitem{b9} Charles Westphal, (18 Nov 2024), ''Feature Selection for Network Intrusion Detection'', https://doi.org/10.48550/arXiv.2411.11603
\bibitem{b10} , , ''The Improved Network Intrusion Detection Techniques Using the Feature Engineering Approach with Boosting Classifiers''
\bibitem{b11} , , ''Why Should We Add Early Exits to Neural Networks''
\bibitem{b12} , , ''Fast yet Safe: Early-Exiting with Risk Control''

\end{thebibliography}

\end{document}
    