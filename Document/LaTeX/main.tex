\documentclass[conference]{IEEEtran}
\IEEEoverridecommandlockouts
% The preceding line is only needed to identify funding in the first footnote. If that is unneeded, please comment it out.
\usepackage[utf8]{inputenc}
\usepackage[T1]{fontenc}
\usepackage{cite}
\usepackage{amsmath,amssymb,amsfonts}
\usepackage{algorithmic}
\usepackage{graphicx}
\usepackage{textcomp}
\usepackage{xcolor}
\usepackage[portuguese]{babel}
\usepackage[colorlinks=true,allcolors=black]{hyperref}
\usepackage[acronym]{glossaries}

% não permite separar acrônimo em duas linhas (só a sigla estar na outra linha)
\newacronymstyle{long-short-br}
{%
  \GlsUseAcrEntryDispStyle{long-short}%
}%
{%
  \GlsUseAcrStyleDefs{long-short}%  
  \renewcommand*{\genacrfullformat}[2]{%
    \glsentrylong{##1}##2~\textup{(\firstacronymfont{\glsentryshort{##1}})}%
  }%
  \renewcommand*{\Genacrfullformat}[2]{%
    \Glsentrylong{##1}##2~\textup{(\firstacronymfont{\glsentryshort{##1}})}%
  }%
  \renewcommand*{\genplacrfullformat}[2]{%
    \glsentrylongpl{##1}##2~\textup{(\firstacronymfont{\glsentryshortpl{##1}})}%
  }%
  \renewcommand*{\Genplacrfullformat}[2]{%
    \Glsentrylongpl{##1}##2~\textup{(\firstacronymfont{\Glsentryshortpl{##1}})}%
  }%
}
\setacronymstyle{long-short-br}


\usepackage{graphicx}
\usepackage{xcolor}
\usepackage{colortbl}
\usepackage{array}
\usepackage{booktabs} % Melhora a aparência das tabelas
\usepackage{soul}
\setuldepth{Berlin}


\definecolor{headercolor}{RGB}{220,220,220} % Cinza claro para cabeçalhos
\definecolor{rowcolor}{RGB}{245,245,245} % Cinza ainda mais claro para linhas alternadas
\newcommand\red[1]{{\textcolor{red}{#1}}}

\sloppy
\begin{document}

\input{acronyms} % importa os acrônimos de acronyms.tex

\title{Revisão da Literatura\\
\vspace{0.4cm}
Tema de pesquisa: Detecção de Intrusão de Redes em CNNs para hardwares limitados\\
\vspace{0.4cm}
{\Large Projeto de Pesquisa Científica\\
Experiência Criativa: Projeto Transformador I\\
Bacharelado em Ciência da Computação -- PUCPR\\
2025}}

\author{
    Turma U -- Equipe 03 \\
    Eduarda Dallagrana,  Luiz Pereira, Kaua Nunes, \red{Nome4 Sobrenome4} \\
    {\tt\small \red{dudadallagrana@gmail.com}, 0009-0008-3781-0134, \red{email3}, \red{email4}}
}
\maketitle


\section{Fundamentação Teórica}

A crescente conectividade expõe novos riscos á toda informação cada vez mais abrangente e sensível, ampliando as oportunidades para ataques cibernéticos. exigindo mecanismos de proteção eficazes. No entanto, a evolução dos ataques, incluindo exploits customizados e ameaças "zero-day", desafia a detecção e resposta rápida, tornando essencial o aprimoramento das estratégias de segurança de forma performática.

Abordagens tradicionais de detecção de intrusão, como árvores de decisão e k-Nearest Neighbors (KNN), apresentam limitações em termos de tempo de processamento e escalabilidade. Entretanto, tecnicas e algoritimos de apreendizarem profunda surgem como uma alternativa promissora, ajustando automaticamente os pesos das entradas sem necessidade de armazenar grandes volumes de dados no processo de identificação. Porém, esses modelos ainda enfrentam desafios computacionais, especialmente quando executados em hardware de uso geral.

Com o crescimento exponencial dos dispositivos de processamento limitado, Iot e a dependência crescente de software de terceiros, o desenvolvimento de sistemas de detecção de intrusão mais rápidos e eficientes são a única defesa dos usuários contra ataques. O desenvolvimento uma solução de alta performace e baixo consumo de hardware capaz de detectar ameaças em tempo real, protegendo tanto dispositivos limitados quanto redes públicas, evita que dados sensíveis de se tornarem prejuízo.

\subsection {Sistemas de Detecção de Intrusão}

    O Instituto Nacional de Padrões e Tecnologia (NIST) define uma intrusão como qualquer ação cujo objetivo seja comprometer a confidencialidade, integridade e disponibilidade de um dispositivo, sistema ou rede, além de violar seus mecanismos de segurança (BACE; MELL, 2001). Dessa forma, um Sistema de Detecção de Intrusão (Intrusion Detection System – IDS) pode ser descrito como um conjunto de ferramentas automatizadas responsáveis por monitorar e analisar eventos em redes ou sistemas computacionais, buscando identificar possíveis atividades maliciosas.
    
    Os IDSs são categorizados em dois tipos principais: (i) Sistema de Detecção de Intrusão em Hospedeiro (Host Intrusion Detection System – HIDS) e (ii) Sistema de Detecção de Intrusão em Rede (Network Intrusion Detection System – NIDS). O HIDS tem como função analisar comportamentos específicos nos dispositivos finais, como o acesso a arquivos, uso de aplicativos e registros de logs. Já o NIDS foca na análise do tráfego de rede, procurando padrões suspeitos de comportamento. Normalmente, um NIDS é posicionado estrategicamente, como próximo a firewalls ou em segmentos críticos da rede (SILVA et al., 2022).
    
    Há três abordagens principais para detecção de intrusão. A detecção baseada em assinatura identifica ataques conhecidos ao compará-los com padrões previamente registrados. Já a detecção por anomalia considera qualquer desvio em relação a um comportamento previamente estabelecido como normal. Diferente da primeira abordagem, essa metodologia permite detectar ameaças ainda desconhecidas. Por fim, a análise de protocolo com estado examina a relação entre requisições e respostas dentro dos protocolos da rede monitorada, permitindo identificar sequências inesperadas e, consequentemente, possíveis intrusões desconhecidas.


\subsection{Aprendizagem de máquina}

    O Aprendizado de Máquina é um processo automatizado onde algoritmos de computador aprendem a identificar padrões em dados (Kelleher et al., 2015). Essa área integra conhecimentos de matemática, estatística e ciência da computação, permitindo análises preditivas avançadas, conforme destacado por Muller e Guido (2017) e Aparicio (2019). Em termos de desempenho dos algoritmos, é importante considerar dois fatores fundamentais: viés e variância. O viés refere-se ao erro introduzido pela simplificação excessiva do modelo, enquanto a variância reflete a sensibilidade do modelo a pequenas variações nos dados de treinamento. Um modelo ideal busca o equilíbrio entre ambos para evitar problemas de underfitting (viés alto) e overfitting (variância alta), garantindo generalização adequada aos dados.

    As tarefas de aprendizado de máquina são tipicamente classificadas em três categorias: aprendizado supervisionado, não supervisionado e semi-supervisionado. No aprendizado supervisionado, o algoritmo aprende padrões a partir de exemplos de entradas e saídas desejadas, fornecidos por um especialista. Assim, ele adquire a capacidade de mapear entradas válidas, de um mesmo contexto, para uma saída correspondente.
    
    Entre as arquiteturas de redes neurais profundas, a Convolutional Neural Network (CNN) se destaca por utilizar camadas de convolução para extrair automaticamente características relevantes do tráfego de rede. Cada camada de convolução aplica filtros (kernels) aos dados de entrada, permitindo identificar padrões anômalos e assinaturas de ataques cibernéticos.

    As camadas de pooling, como Max Pooling e Average Pooling, seguem as camadas de convolução e têm a função de reduzir a dimensionalidade dos mapas de características, diminuindo assim a carga computacional e ajudando a evitar o overfitting. O Max Pooling seleciona o valor máximo dentro de uma região do mapa de características, enquanto o Average Pooling calcula a média dos valores, preservando informações essenciais.
    
    Após a extração de características, as camadas flatten transformam os mapas multidimensionais em um vetor unidimensional, que é então alimentado nas camadas totalmente conectadas (fully connected layers). Essas camadas finais realizam a classificação, diferenciando tráfego legítimo de possíveis tentativas de intrusão com base nos padrões aprendidos.
    
    As CNNs têm se tornado uma abordagem promissora para a detecção de intrusão em redes, devido à sua capacidade de analisar grandes volumes de dados e identificar ameaças de forma automática e eficiente.



\subsection{Trabalhos Relacionados}

    (BISWAS, 2018) realiza a comparação entre diversas abordagens de aprendizado de máquina, diferentes misturas de features para os algoritmos de classificação, abordando as  vantagens e desvantagens do uso de cada uma das técnicas e ferramentas, realizando um estudo
    comparativo da abordagem de classificadores

    (AHMAD et al., 2020) Realiza um estudo de sistemas de detecção de intrusão em redes utilizando artigos e conhecimentos recentes como o progresso do uso de Aprendizado de máquina e inteligência artificial no campo dos sistemas de detecção de intrusão, bem como uma análise criteriosa de performance, bases utilizadas e precisão.


\begin{table}[!htb]
    \centering
    \setlength{\tabcolsep}{1pt}
    \caption{Distribuição do número de artigos de conferência e de periódico por estrato Qualis.}
    \label{tab:sumario-qualis}
    
    \vspace{1mm}
    
    \begin{tabular}{ccc}
    \toprule
    \begin{tabular}[c]{c}\textbf{Estrato}\\\textbf{Qualis}\end{tabular} & \begin{tabular}[c]{c}\textbf{Artigos de}\\\textbf{Conferência}\end{tabular} & \begin{tabular}[c]{c}\textbf{Artigos de}\\\textbf{Periódico}\end{tabular} \\ \midrule
    A1     & $0$           & $0$ \\
    A2     & $0$           & $0$ \\
    A3     & $0$           & $0$ \\
    A4     & $0$           & $0$ \\
    Outro  & $0$           & $0$ \\ \bottomrule        
    \end{tabular}
\end{table}



\begin{table}[!htb]
    \centering
    \setlength{\tabcolsep}{4pt}
    \caption{Distribuição do número de artigos por período de publicação.}
    \label{tab:sumario-periodo}
    
    \vspace{1mm}
    
    \begin{tabular}{cc}
    \toprule
    \begin{tabular}[c]{c}\textbf{Ano de}\\\textbf{Publicação}\end{tabular} & \textbf{Artigos} \\ \midrule
    $2021$ - $2025$     & $0$ \\
    $2016$ - $2020$     & $2$ \\
    $2011$ - $2015$     & $0$ \\
    $1900$ - $2010$     & $0$ \\ \bottomrule        
    \end{tabular}
\end{table}


  \begin{table*}[!htb]
    \centering
    \caption{Comparação de Trabalhos no Estado-da-Arte}
    \renewcommand{\arraystretch}{1.3} % Ajuste de espaçamento entre linhas
    \resizebox{\textwidth}{!}{ % Ajusta automaticamente a largura da tabela
        \begin{tabular}{ccccccc}
            \toprule
            \textbf{Autor(es) / Ano} & \textbf{Tema Principal} & \textbf{Método} & \textbf{Base / Amostra} & \textbf{Resultados} & \textbf{Protocolo / Ferramentas} & \textbf{Crítica e Limitações} \\
            \midrule
           Biswas (2018) & Comparação de classificadores para detecção de intrusão &Algoritmos de aprendizado de máquina (Random Forest, SVM, KNN, etc.) & Dataset NSL-KDD & Random Forest: 89.83\% SVM: 82.23\% KNN: 84.16\% de acurácia & Weka e Python & Desempenho variável entre classificadores; necessidade de otimização de features \\  
           Ahmad et al. (2020) & Uso de ML e IA em sistemas de detecção de intrusão & Revisão sistemática e análise de performance & KDD, CICIDS, UNSW-NB15 & Varia entre 85\% e 99\% de acurácia dependendo do modelo e dataset & Análise comparativa de artigos científicos & Pouca generalização e falta de padronização na avaliação dos modelos \\   
            \bottomrule
        \end{tabular}
    }
    
    \label{tab:sota-overview}
    \end{table*}
    

\section{Considerações Finais}


    Este estudo analisou a detecção de intrusão em redes usando técnicas de aprendizado profundo, com foco em Redes Neurais Convolucionais (CNNs). A pesquisa mostrou que métodos tradicionais, como os baseados em assinaturas, têm limitações na identificação de ataques desconhecidos e requerem intervenção manual na extração de características. As CNNs, por outro lado, podem automatizar a extração de padrões importantes dos dados de tráfego, melhorando a precisão e a eficiência dos sistemas.
    
    Os benefícios das CNNs incluem aprendizado contínuo e capacidade de identificar ameaças complexas. Contudo, desafios como alto custo computacional e necessidade de muitos dados rotulados ainda dificultam sua adoção ampla. A maioria dos estudos usa conjuntos de dados específicos, o que pode limitar a aplicabilidade em diferentes cenários.
    
    Futuros trabalhos podem combinar CNNs com outras técnicas de aprendizado de máquina para melhorar a detecção e reduzir falsos positivos. Também, abordagens de aprendizado semi-supervisionado ou auto-supervisionado podem diminuir a dependência de dados rotulados.
    
    Para desenvolver esta pesquisa, a abordagem será quantitativa, avaliando a performance das CNNs através de métricas como acurácia e taxa de falsos alarmes, e realizando estudos de caso com bases de dados variadas. O objetivo é contribuir para a detecção inteligente de ameaças e aumentar a eficiência e adaptabilidade dos sistemas de segurança cibernética.
    
\begin{thebibliography}{00}

\bibitem{b1}BISWAS, Saroj Kr. Intrusion Detection Using Machine Learning: A Comparison Study. 2018. NIT Silchar.
\bibitem{b2}AHMAD, Zeeshan et al. Network intrusion detection system: A systematic study of machine learning and deep learning approaches. 2020. UniversitiMalaysia Sarawak.
\bibitem{b3} Nathan Shone; Tran Nguyen Ngoc; Vu Dinh Phai; Qi Shi  (February 2018), ''A Deep Learning Approach to Network Intrusion Detection'', Australian Journal of Mechanical Engineering, IEEE Transactions on Emerging Topics in Computational Intelligence ( Volume: 2, Issue: 1, February 2018)
\bibitem{b4} Springer, Berlin, Heidelberg, ''Early Stopping - But When?'' Part of the book series: Lecture Notes in Computer Science ((LNCS,volume 1524))
\bibitem{b5} Hanan Hindy,Robert Atkinson, Christos Tachtatzis,Jean-Noël Colin, Ethan Bayne, and Xavier Bellekens, (14 October 2020), ''Utilising Deep Learning Techniques for Effective Zero-Day Attack Detection'' ELECTRONIC COMMERCE RESEARCH AND APPLICATIONS, doi: https://doi.org/10.3390/electronics9101684
\bibitem{b6} Akinul Islam Jony, Arjun Kumar Bose Arnob, (August 8, 2024), ''Securing the Internet of Things: Evaluating Machine Learning Algorithms for Detecting IoT Cyberattacks Using CIC-IoT2023 Dataset'', DOI: 10.5815/ijitcs.2024.04.04
\bibitem{b7} , , ''Robust Botnet Detection Approach for Known and Unknown Attacks in IoT Networks Using Stacked Multi-classifier and Adaptive Thresholding''
\bibitem{b8} , , ''Securing the Internet of Things: Evaluating Machine Learning Algorithms for Detecting IoT Cyberattacks Using CIC-IoT2023 Dataset''
\bibitem{b9} Charles Westphal, (18 Nov 2024), ''Feature Selection for Network Intrusion Detection'', https://doi.org/10.48550/arXiv.2411.11603
\bibitem{b10} , , ''The Improved Network Intrusion Detection Techniques Using the Feature Engineering Approach with Boosting Classifiers''
\bibitem{b11} , , ''Why Should We Add Early Exits to Neural Networks''
\bibitem{b12} , , ''Fast yet Safe: Early-Exiting with Risk Control''

\end{thebibliography}

\end{document}
    